\input{../common/plantilla-ejercicio.tex}
\usepackage{eurosym}





\renewcommand{\hmwkTitle}{Ejercicios de \textit{Microsoft Word}}
\renewcommand{\hmwkClass}{TICO1}


\usepackage{blindtext}

\begin{document}

% \maketitle

% ----------------------------------------------------------------------------------------
%	TABLE OF CONTENTS
% ----------------------------------------------------------------------------------------

% \setcounter{tocdepth}{1} % Uncomment this line if you don't want subsections listed in the ToC

\primerapagina

\setlength{\parindent}{0em}
\setlength{\parskip}{1em}

\section{Objetivo de la práctica}
Se pretende que el alumno ejercite las funcionalidades de \textit{Microsoft Word} necesarias para automatizar ciertas tareas de la edición, lo que posibilitará escribir documentos gran extensión.



\newcommand{\urlagithub}[1]{%
  \enlace{https://alvarogonzalezsotillo.github.io/tico1-bach1/apuntes/5/textos/#1}{#1}%
  \hspace{0.5em}%
  \textattachfile[color=0 0 1]{./textos/#1}{({\scriptsize adjunto}\hspace{0.2em}\includegraphics[height=1em,width=1em]{attach-icon.pdf})}%
}

\begin{homeworkProblem}[: Documento técnico (5 puntos)]
  Observa los siguientes documentos:
  \begin{itemize}
  \item \urlagithub{rfc1178-nombre-ordenador.txt}
  \item \urlagithub{rfc2606-nombres-dns-reservados.txt}
  \item \urlagithub{rfc2664-preguntas-internet.txt}
  \item \urlagithub{rfc3271-internet-para-todos.txt}      
  \end{itemize}

  Son documentos formateados para ser impresos con una impresora antigua. Por tanto, son simples ficheros de texto, pero incluyen muchas de las características de un documento profesional: Cabecera y pie de página,  números de página,  tabla de contenidos,  notas al pie,  referencias cruzadas,  títulos de primer y segundo nivel. Elige uno de los ficheros y actualiza su formato a \textit{Microsoft Word}.

  \begin{Aviso}[Qué se valorará]

    \begin{itemize}
    \item Que incluya (aunque el original no lo tenga)
      \begin{itemize}
      \item Tabla de contenidos
      \item Notas al pie
      \item Referecias cruzadas (a un párrafo, imagen, capítulo...)
      \end{itemize}
    \item No es necesario que coincida el número de páginas entre la versión \textit{Word} y la orignal    
    \item Respetar los títulos y sus niveles
    \item Respetar la cabecera y pie de página con la información original, excepto:
      \begin{itemize}
      \item Números de página, que pueden cambiar si cambia la paginación
      \item La fecha, que será la fecha de última modificación del fichero
      \end{itemize}
    \end{itemize}
  \end{Aviso}
\end{homeworkProblem}

\newpage

\begin{homeworkProblem}[: Texto literario (5 puntos)]
  Formatea una historia corta como si fuera una pequeña novela. Se proponen los siguientes cuentos de \enlace{https://es.wikipedia.org/wiki/Cosmicismo}{horror cósmico}. 
  \begin{itemize}
  \item \urlagithub{la-hoya-de-las-brujas.txt}
  \item \urlagithub{las-ratas-del-cementerio.txt}
  \end{itemize}
  También puedes elegir otras historias cortas. En ese caso, muestra al profesor la historia en formato texto antes de empezar a formatearla.

  \begin{Aviso}[Qué se valorará]

    \begin{itemize}
    \item Que incluya (aunque originalmente no lo tenga)
      \begin{itemize}
      \item Portada
      \item Tabla de contenidos
      \item Notas al pie
      \item Referecias cruzadas (a un párrafo, imagen, capítulo...)
      \item Alguna ilustración
      \end{itemize}
    \item Que incluya números de página
    \item Las ilustraciones, colores y tipografía deben contribuir al tema de la historia (horror cósmico o el tema elegido por el alumno)
    \end{itemize}
  \end{Aviso}
  
\end{homeworkProblem}




\section{Instrucciones de entrega}
\begin{itemize}
\item El ejercicio se realizará y entregará de manera individual.
  \begin{itemize}
  \item Solo se admiten trabajos en pareja, si en clase es necesario compartir ordenador.
  \end{itemize}
\item Entrega tu trabajo en un fichero \texttt{zip} con los dos documentos de \textit{Microsoft Word}
  \begin{itemize}
  \item El profesor necesita los ficheros \texttt{docx} o \texttt{doc} para corregir la práctica, no lo envíes en otro formato como \texttt{pdf}
  \end{itemize}

\item Sube el documento a \enlace{http://aulavirtual2.educa.madrid.org/mod/assignment/view.php?id=1156675}{la tarea correspondiente en el aula virtual}
\item Presta atención al plazo de entrega (con fecha y hora).
  
\end{itemize}
\end{document}




